

Le programme MASTA est une simulation multi-agents d'allocation de
tâches pour la récolte de ressources. Pour cela, il simule différentes
sortes d'agents~: les huttes, les humains et les moutons. De plus, le
sol par l'intermédiaire des \og{}patch\fg{} représente dans
différentes proportions de l'herbe, de l'eau, des arbres et des baies.

\section{Les agents humains}



\section{Les huttes}

Les huttes jouent un rôle crucial dans masta. Elles organisent la
répartition des tâches entre les agents \og{}humains\fg{}. Le but pour
chaque hutte est de récolter le maximum de ressources possibles de la
façon la mieux répartie possible. Trois types de huttes ont été
implémenter, représentant chacun un type d'allocation différents.

\begin{minipage}[H]{0.1\linewidth}
  \begin{figure}[H]
    \begin{center}
      \includegraphics[width=0.5\textwidth]{./img/hut_static}
    \end{center}
  \end{figure}
\end{minipage}
\begin{minipage}[H]{0.9\linewidth}
  Les huttes de type \og{}static\fg{} alloue une tâche à chaque agent
  humain une fois pour toute.
\end{minipage}

\begin{minipage}[H]{0.1\linewidth}
  \begin{figure}[H]
    \begin{center}
      \includegraphics[width=0.5\textwidth]{./img/hut_min_before}
    \end{center}
  \end{figure}
\end{minipage}
\begin{minipage}[H]{0.9\linewidth}
  Les huttes de type \og{}min before\fg{}
\end{minipage}

\begin{minipage}[H]{0.1\linewidth}
  \begin{figure}[H]
    \begin{center}
      \includegraphics[width=0.5\textwidth]{./img/hut_affinity}
    \end{center}
  \end{figure}
\end{minipage}
\begin{minipage}[H]{0.9\linewidth}
  Les huttes de types\og{}affinity\fg{}
\end{minipage}


\section{Section 2}
\section{Section 3}
